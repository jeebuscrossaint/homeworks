\documentclass{article}
\usepackage{amsmath}
\usepackage{amssymb}

\author{Amarnath Patel}
\title{Exam 3: COT 2000}
\date{\today}

\begin{document}
\maketitle

\section{Question 1}
(a) The first four terms:
\[a_0 = 5, a_1 = 17, a_2 = 29, a_3 = 41\]

(b) To show convergence:
- \[ \lim_{n \to \infty} a_n = \lim_{n \to \infty} 5 + \lim_{n \to \infty} \left(\frac{1}{2}\right)^n \]
- \[ \lim_{n \to \infty} \left(\frac{1}{2}\right)^n = 0 \]
- \[ \lim_{n \to \infty} a_n = 5 + 0 = 5 \]
ans: \( a_n = 5 + \left(\frac{1}{2}\right)^n \) converges to 5 as \( n \) approaches infinity.

\section{Question 2}
(a) $b_n = 2n, \text{ for } n \geq 1$

(b) $b_n = 3(-1)^{n+1}, \text{ for } n \geq 1$

(c) $b_n = \frac{n}{2n+1}, \text{ for } n \geq 1$

\section{Question 3}
(a) $\sum_{k=2}^7 (k+3) = 5 + 6 + 7 + 8 + 9 + 10 = 45$

(b) $\prod_{k=2}^4 (k+1)^2 = 3^2 \cdot 4^2 \cdot 5^2 = 3600$

\section{Question 4}
Let \( X = \{1, 3, 5, 7, 9\} \) and \( Y = \{2, 4, 6, 8, 10\} \).

(a) Define \( f : X \to Y \) by:
\[
f(1) = 8, \quad
f(3) = 6, \quad
f(5) = 4, \quad
f(7) = 8, \quad
f(9) = 10.
\]

Is \( f \) one-to-one? Is \( f \) onto? Explain your answers.

\begin{itemize}
    \item \( f \) is \textbf{not one-to-one} because \( f(1) = 8 \) and \( f(7) = 8 \), implying \( f \) is not injective.
    \item \( f \) is \textbf{not onto} because \( 2 \in Y \) is not mapped by any element in \( X \), implying \( f \) is not surjective.
\end{itemize}

(b) Define \( g : X \to Y \) by:
\[
g(1) = 2, \quad
g(3) = 4, \quad
g(5) = 6, \quad
g(7) = 8, \quad
g(9) = 10.
\]

Is \( g \) one-to-one? Is \( g \) onto? Explain your answers.

\begin{itemize}
    \item \( g \) is \textbf{one-to-one} because all values are distinct, implying \( g \) is injective.
    \item \( g \) is \textbf{onto} because every element in \( Y \) is mapped by some element in \( X \), implying \( g \) is surjective.
\end{itemize}

\section{Question 5}
(a) F is not one-to-one (e is repeated). F is not onto (g is not in the range).

(b) G is one-to-one. G is onto.

(c) To make F and G one-to-one correspondences:
F: a→e, b→f, c→g (remove d)
G: a→e, b→f, c→g, d→h (add h to Y)

\section{Question 6}
(a) h(x) = 2x + 1

i. One-to-one proof:
\[h(x_1) = h(x_2) \implies 2x_1 + 1 = 2x_2 + 1 \implies x_1 = x_2\]

ii. Onto proof:
For any $y \in \mathbb{Z}$, $x = \frac{y-1}{2} \in \mathbb{Z}$ satisfies $h(x) = y$

(b) $g(x) = x^2 - 1$

i. Not one-to-one: g(1) = g(-1) = 0

ii. Not onto: g(x) $\geq$ -1 for all x, so y < -1 has no preimage

\end{document}
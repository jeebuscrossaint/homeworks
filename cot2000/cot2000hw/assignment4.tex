\documentclass{article}
\usepackage{amsmath}
\begin{document}

\author{Amarnath Patel}
\date{\today}
\title{Assignment 4}
\maketitle

\section*{Question 1}

\textbf{(a) Sequence: } $a_k = \frac{2k}{5 + k}$ \text{ for all integers } $k \geq 1$

First four terms:
\begin{align*}
a_1 &= \frac{2 \cdot 1}{5 + 1} = \frac{2}{6} = \frac{1}{3} \\
a_2 &= \frac{2 \cdot 2}{5 + 2} = \frac{4}{7} \\
a_3 &= \frac{2 \cdot 3}{5 + 3} = \frac{6}{8} = \frac{3}{4} \\
a_4 &= \frac{2 \cdot 4}{5 + 4} = \frac{8}{9}
\end{align*}

\textbf{(b) Sequence: } $b_j = \frac{4 - j}{4 + j}$ \text{ for all integers } $j \geq 1$

First four terms:
\begin{align*}
b_1 &= \frac{4 - 1}{4 + 1} = \frac{3}{5} \\
b_2 &= \frac{4 - 2}{4 + 2} = \frac{2}{6} = \frac{1}{3} \\
b_3 &= \frac{4 - 3}{4 + 3} = \frac{1}{7} \\
b_4 &= \frac{4 - 4}{4 + 4} = \frac{0}{8} = 0
\end{align*}

\section*{Question 2}

\textbf{(a) Sequence: } $e_m = 2 + \left(\frac{1}{3}\right)^m$ \text{ for all integers } $m \geq 0$

First four terms:
\begin{align*}
e_0 &= 2 + \left(\frac{1}{3}\right)^0 = 2 + 1 = 3 \\
e_1 &= 2 + \left(\frac{1}{3}\right)^1 = 2 + \frac{1}{3} = \frac{7}{3} \\
e_2 &= 2 + \left(\frac{1}{3}\right)^2 = 2 + \frac{1}{9} = \frac{19}{9} \\
e_3 &= 2 + \left(\frac{1}{3}\right)^3 = 2 + \frac{1}{27} = \frac{55}{27}
\end{align*}

\textbf{(b) Showing convergence:}

As $m \to \infty$, $\left(\frac{1}{3}\right)^m \to 0$. Therefore,
\[
\lim_{m \to \infty} e_m = \lim_{m \to \infty} \left(2 + \left(\frac{1}{3}\right)^m\right) = 2 + 0 = 2
\]

Thus, the sequence $e_m$ converges to 2 as $m$ approaches infinity.

\section*{Question 3}

\textbf{(a) Sequence: } $a_n = (-1)^{n+1}$ for all integers $n \geq 1$

Explicit formula:
\[
a_n = \begin{cases}
1 & \text{if } n \text{ is odd} \\
-1 & \text{if } n \text{ is even}
\end{cases}
\]

\textbf{(b) Sequence: } $b_n = (-1)^{n+1} \cdot \left\lfloor \frac{n}{2} \right\rfloor$ for all integers $n \geq 1$

Explicit formula:
\[
b_n = \begin{cases}
0 & \text{if } n \equiv 0 \pmod{2} \\
-n & \text{if } n \equiv 1 \pmod{4} \\
n-1 & \text{if } n \equiv 3 \pmod{4}
\end{cases}
\]

\textbf{(c) Sequence: } $c_n = \frac{n+1}{2}$ for all integers $n \geq 1$

Explicit formula:
\[
c_n = \frac{2n - 1}{2}
\]

\section*{Question 4}

\textbf{Summation: }
\[
\sum_{k=1}^{6} (k+2) = (1+2) + (2+2) + (3+2) + (4+2) + (5+2) + (6+2)
\]
Calculating each term:
\[
= 3 + 4 + 5 + 6 + 7 + 8 = 33
\]

\textbf{Product: }
\[
\prod_{k=3}^{5} k^2 = 3^2 \cdot 4^2 \cdot 5^2
\]
Calculating each square:
\[
= 9 \cdot 16 \cdot 25 = 3600
\]

\section*{Question 5}

\textbf{Proposition:} Prove using mathematical induction that for all integers $n \geq 0$,
\[ P(n): \quad 1 + 2 + 4 + 8 + \cdots + 2n = 2^{n+1} - 1 \]

\textbf{(a) Base Case:} $P(0)$

For $n = 0$,
\[ 1 = 2^{0+1} - 1 = 2 - 1 = 1 \]
So, $P(0)$ holds true.

\textbf{(b) Inductive Step:} Assume $P(k)$ is true for some integer $k \geq 0$, i.e.,
\[ 1 + 2 + 4 + 8 + \cdots + 2k = 2^{k+1} - 1 \]

Now, prove $P(k+1)$:
\[ 1 + 2 + 4 + 8 + \cdots + 2k + 2^{k+1} = 2^{(k+1)+1} - 1 \]

Starting from the left-hand side,
\[ 1 + 2 + 4 + 8 + \cdots + 2k + 2^{k+1} = (2^{k+1} - 1) + 2^{k+1} \]
\[ = 2^{k+1} + 2^{k+1} - 1 \]
\[ = 2 \cdot 2^{k+1} - 1 \]
\[ = 2^{k+2} - 1 \]

$P(n)$ is true for all integers $n \geq 0$.

\end{document}

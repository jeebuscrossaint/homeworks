\documentclass{article}
\usepackage{amsmath}

\title{Assignment 2}
\author{Amarnath Patel}
\date{\today}

\begin{document}
\maketitle
\begin{enumerate}
\item
\begin{enumerate}
\item \(h \land w \land \sim s\)
\item \(\sim w \land h \land s\)
\item \(\sim h \land \sim w \land \sim s\)
\item \(\sim w \land \sim s \land h\)
\item \(w \land \sim (h \land s)\)
\end{enumerate}

\item
\begin{enumerate}
\item 

\begin{tabular}{|c c|c|}
\hline
p & q & \(\sim (p \land q) \lor (p \lor q)\) \\ [0.5ex] 
\hline
T & T & T \\ 
T & F & T \\
F & T & T \\
F & F & T \\
\hline
\end{tabular}

\item 

\begin{tabular}{|c c c|c|}
\hline
p & q & r & \(p \land (\sim q \lor r)\) \\ [0.5ex] 
\hline
T & T & T & T \\ 
T & T & F & F \\
T & F & T & T \\
T & F & F & T \\
F & T & T & F \\
F & T & F & F \\
F & F & T & F \\
F & F & F & F \\
\hline
\end{tabular}

\end{enumerate}

\item
\begin{enumerate}
\item \(p \lor (p \land q)\) and \(q\) are not logically equivalent because their truth tables are not the same.
\item \(\sim (p \land q)\) and \(\sim p \land \sim q\) are logically equivalent because their truth tables are the same.
\item \(p \land (q \lor r)\) and \((p \land q) \lor (p \land r)\) are logically equivalent because their truth tables are the same.
\end{enumerate}

\item
\begin{enumerate}
\item \(\sim (-2 < x < 6) \rightarrow x \leq -2 \lor x \geq 6\)
\item \(\sim (-9 < x < 2) \rightarrow x \leq -9 \lor x \geq 2\)
\item \(\sim (x < 2 \lor x > 6) \rightarrow x \geq 2 \land x \leq 6\)
\item \(\sim (x \leq -1 \lor x > 1) \rightarrow x > -1 \land x \leq 1\)
\item \(\sim (0 > x \geq -4) \rightarrow x \leq 0 \land x < -4\)
\end{enumerate}

\item
\begin{enumerate}
\item \((p \land q) \lor (\sim p \lor (p \land \sim q))\) is a tautology.
\item \((p \land \sim q) \land (\sim p \lor q)\) is a contradiction.
\item \(((\sim p \land q) \land (q \land r)) \lor \sim q\) is neither a tautology nor a contradiction.
\end{enumerate}

\item
\begin{enumerate}
\item Let \(b\) represent "Bob is a double math and computer science major" and \(a\) represent "Ann is a math major". Then the statement is \(b \land a \land \sim a\).
\item The statement is \(\sim (b \land a) \land a \land b\). The two statements are not logically equivalent.
\end{enumerate}

\item \((p \oplus q) \oplus r \equiv p \oplus (q \oplus r)\) by the associative law of exclusive or.

\item
\begin{enumerate}
\item \((p \land \sim q) \lor (p \land q) \equiv p \land (\sim q \lor q)\) by distributive law.
\item \(p \land (\sim q \lor q) \equiv p \land t\) by law of excluded middle.
\item \(p \land t \equiv p\) by identity law.
\end{enumerate}

\item \((p \land \sim q) \lor p \equiv p\) by absorption law.

\item \(\sim ((\sim p \land q) \lor (\sim p \land \sim q)) \lor (p \land q) \equiv p\) by De Morgan's law and law of excluded middle.

\end{enumerate}

\end{document}
\documentclass{article}
\usepackage{amsmath}

\begin{document}

\title{Assignment 5, Questions Chosen: 3, 13, 22, 29, 31}
\author{Amarnath Patel}
\date{\today}


\maketitle

\section{Question 3}
\begin{align*}
    \text{Smallest class size} &= 26 \times 26 + 1 \\
    &= 676 + 1 \\
    &= 677
    \end{align*}
    
    \text{Where:}
    \begin{itemize}
        \item 26 = \text{number of possible first initials}
        \item 26 = \text{number of possible last initials}
        \item 1 \text{ is added to exceed the maximum unique combinations}
    \end{itemize}

\section{Question 13}
\begin{align*}
    \text{Number of integers} &= 9 \times 9 \times 9 \\
    &= 9^3 \\
    &= 729
    \end{align*}
    
    \text{Where:}
    \begin{itemize}
        \item 9 \text{ choices for hundreds place } (1-9, \text{excluding } 7)
        \item 9 \text{ choices for tens place } (0-9, \text{excluding } 7)
        \item 9 \text{ choices for ones place } (0-9, \text{excluding } 7)
    \end{itemize}

\section{Question 22}
\begin{align*}
    \text{Number of ways} &= 1000 \times 999 \times 998 \\
    &= 1000 \times 999 \times 998 \\
    &= 997,002,000
    \end{align*}
    
    \text{This is equivalent to the permutation:}
    \begin{equation}
    P(1000, 3) = \frac{1000!}{(1000-3)!} = \frac{1000!}{997!}
    \end{equation}

\section{Question 29}
\begin{align*}
    \text{Number of seating arrangements} &= 5! \times 2 \times 3! \times 3! \\
    &= 120 \times 2 \times 6 \times 6 \\
    &= 8,640
    \end{align*}
    
    \text{Where:}
    \begin{itemize}
        \item 5! \text{ : arrangements of the other 5 people relative to the president}
        \item 2 \text{ : ways to alternate math and CS majors (MCMCMC or CMCMCM)}
        \item 3! \text{ : permutations of math majors among their positions}
        \item 3! \text{ : permutations of CS majors among their positions}
    \end{itemize}

\section{Question 31}
\begin{align*}
    \text{Number of possible committees} &= \binom{10}{3} \times \binom{25}{4} \\[6pt]
    &= \frac{10!}{3!(10-3)!} \times \frac{25!}{4!(25-4)!} \\[6pt]
    &= \frac{10!}{3!7!} \times \frac{25!}{4!21!} \\[6pt]
    &= 120 \times 12,650 \\[6pt]
    &= 1,518,000
    \end{align*}
    
    \text{Where:}
    \begin{itemize}
        \item $\binom{10}{3}$: ways to choose 3 faculty members from 10
        \item $\binom{25}{4}$: ways to choose 4 students from 25
    \end{itemize}

\end{document}
\documentclass{article}
\usepackage{amsmath}
\usepackage{amssymb}

\begin{document}

\title{Exam 2}
\author{Amarnath Patel Z23614792}
\date{\today}

\maketitle

\subsection*{1. Truth Table for $(p \wedge r) \leftrightarrow (q \wedge r)$}

\begin{center}
\begin{tabular}{|c|c|c|c|c|c|c|}
\hline
$p$ & $q$ & $r$ & $p \wedge r$ & $q \wedge r$ & $(p \wedge r) \leftrightarrow (q \wedge r)$ \\
\hline
T & T & T & T & T & T \\
T & T & F & F & F & T \\
T & F & T & T & F & F \\
T & F & F & F & F & T \\
F & T & T & F & T & F \\
F & T & F & F & F & T \\
F & F & T & F & F & T \\
F & F & F & F & F & T \\
\hline
\end{tabular}
\end{center}

\subsection*{2. Rewriting Statement Forms}

For $p \vee \neg q \rightarrow r$:

(a) Using $p \rightarrow q \equiv \neg p \vee q$:
    $\neg(p \vee \neg q) \vee r$
    
(b) Using $p \vee q \equiv \neg(\neg p \wedge \neg q)$:
    $\neg(\neg(\neg p \wedge q) \wedge \neg r)$

\subsection*{3. Negations, Contrapositives, Converses, and Inverses}

(a) Original: If P is a triangle, then P is an equilateral triangle.
   
   Negation: P is a triangle and P is not an equilateral triangle.
   
   Contrapositive: If P is not an equilateral triangle, then P is not a triangle.
   
   Converse: If P is an equilateral triangle, then P is a triangle.
   
   Inverse: If P is not a triangle, then P is not an equilateral triangle.

(b) Original: If John is Mike's brother, then Sarah is his sister and Paul is his cousin.
   
   Negation: John is Mike's brother and either Sarah is not his sister or Paul is not his cousin.
   
   Contrapositive: If Sarah is not his sister or Paul is not his cousin, then John is not Mike's brother.
   
   Converse: If Sarah is his sister and Paul is his cousin, then John is Mike's brother.
   
   Inverse: If John is not Mike's brother, then Sarah is not his sister or Paul is not his cousin.

\subsection*{4. Identifying Converse Error}

The argument form that exhibits the converse error is (a):

If Alice finished her project, then she will present it.
Alice presented her project.
Therefore, Alice finished her project.

This is the converse of the original implication and is not a valid logical conclusion.

\subsection*{5. Validity of Argument Forms}

(a) $(p \vee q) \rightarrow \neg r$, $\neg p \wedge q$, $q \rightarrow p$ $\therefore \neg r$

\begin{center}
\begin{tabular}{|c|c|c|c|c|c|c|}
\hline
$p$ & $q$ & $r$ & $(p \vee q) \rightarrow \neg r$ & $\neg p \wedge q$ & $q \rightarrow p$ & $\neg r$ \\
\hline
T & T & T & F & F & T & F \\
T & T & F & T & F & T & T \\
T & F & T & T & F & T & F \\
T & F & F & T & F & T & T \\
F & T & T & F & T & F & F \\
F & T & F & T & T & F & T \\
F & F & T & T & F & T & F \\
F & F & F & T & F & T & T \\
\hline
\end{tabular}
\end{center}

This argument form is invalid. There is a row (F, T, T) where all premises are true but the conclusion is false.

(b) $p \rightarrow q$, $r \rightarrow q$ $\therefore (p \vee r) \rightarrow q$

\begin{center}
\begin{tabular}{|c|c|c|c|c|c|}
\hline
$p$ & $q$ & $r$ & $p \rightarrow q$ & $r \rightarrow q$ & $(p \vee r) \rightarrow q$ \\
\hline
T & T & T & T & T & T \\
T & T & F & T & T & T \\
T & F & T & F & F & F \\
T & F & F & F & T & F \\
F & T & T & T & T & T \\
F & T & F & T & T & T \\
F & F & T & T & F & F \\
F & F & F & T & T & T \\
\hline
\end{tabular}
\end{center}

This argument form is valid. In every row where both premises are true, the conclusion is also true.

\subsection*{6. Predicate Q(x): $x^2 < 2x$}

(a) Q(1): $1^2 < 2(1)$ is true (1 < 2)
    Q(2): $2^2 < 2(2)$ is false (4 > 4)
    Q(0): $0^2 < 2(0)$ is false (0 = 0)
    Q(-1): $(-1)^2 < 2(-1)$ is false (1 > -2)
    Q(3): $3^2 < 2(3)$ is false (9 > 6)

(b) Truth set for domain R: (0, 2)

(c) Truth set for domain R+: (0, 2)

\subsection*{7. Truth Set for R(x): $x^2 - 5x + 6 = 0$}

The correct answer is (a) {2, 3}.

Explanation: We can factor the equation as $(x-2)(x-3) = 0$. The solutions are x = 2 and x = 3.

\subsection*{8. Formal Negations}

(a) $\exists$ birds x such that x cannot fly.
(b) $\exists$ cars c such that c does not have wheels.
(c) $\forall$ buildings b, b is not over 100 stories tall.
(d) $\forall$ trees t, t is not over 100 years old.

\subsection*{9. Representation of "Every engineering major is also a physics major"}

The correct answer is (a) $\forall s \in D, M(s) \Rightarrow S(s)$

This statement reads as "For all students s in D, if s is an engineering major, then s is a physics major," which correctly represents the given statement.

\end{document}